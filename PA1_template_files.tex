\PassOptionsToPackage{unicode=true}{hyperref} % options for packages loaded elsewhere
\PassOptionsToPackage{hyphens}{url}
%
\documentclass[]{article}
\usepackage{lmodern}
\usepackage{amssymb,amsmath}
\usepackage{ifxetex,ifluatex}
\usepackage{fixltx2e} % provides \textsubscript
\ifnum 0\ifxetex 1\fi\ifluatex 1\fi=0 % if pdftex
  \usepackage[T1]{fontenc}
  \usepackage[utf8]{inputenc}
  \usepackage{textcomp} % provides euro and other symbols
\else % if luatex or xelatex
  \usepackage{unicode-math}
  \defaultfontfeatures{Ligatures=TeX,Scale=MatchLowercase}
\fi
% use upquote if available, for straight quotes in verbatim environments
\IfFileExists{upquote.sty}{\usepackage{upquote}}{}
% use microtype if available
\IfFileExists{microtype.sty}{%
\usepackage[]{microtype}
\UseMicrotypeSet[protrusion]{basicmath} % disable protrusion for tt fonts
}{}
\IfFileExists{parskip.sty}{%
\usepackage{parskip}
}{% else
\setlength{\parindent}{0pt}
\setlength{\parskip}{6pt plus 2pt minus 1pt}
}
\usepackage{hyperref}
\hypersetup{
            pdftitle={Repro. Research Project 1},
            pdfauthor={slauzon},
            pdfborder={0 0 0},
            breaklinks=true}
\urlstyle{same}  % don't use monospace font for urls
\usepackage[margin=1in]{geometry}
\usepackage{color}
\usepackage{fancyvrb}
\newcommand{\VerbBar}{|}
\newcommand{\VERB}{\Verb[commandchars=\\\{\}]}
\DefineVerbatimEnvironment{Highlighting}{Verbatim}{commandchars=\\\{\}}
% Add ',fontsize=\small' for more characters per line
\usepackage{framed}
\definecolor{shadecolor}{RGB}{248,248,248}
\newenvironment{Shaded}{\begin{snugshade}}{\end{snugshade}}
\newcommand{\AlertTok}[1]{\textcolor[rgb]{0.94,0.16,0.16}{#1}}
\newcommand{\AnnotationTok}[1]{\textcolor[rgb]{0.56,0.35,0.01}{\textbf{\textit{#1}}}}
\newcommand{\AttributeTok}[1]{\textcolor[rgb]{0.77,0.63,0.00}{#1}}
\newcommand{\BaseNTok}[1]{\textcolor[rgb]{0.00,0.00,0.81}{#1}}
\newcommand{\BuiltInTok}[1]{#1}
\newcommand{\CharTok}[1]{\textcolor[rgb]{0.31,0.60,0.02}{#1}}
\newcommand{\CommentTok}[1]{\textcolor[rgb]{0.56,0.35,0.01}{\textit{#1}}}
\newcommand{\CommentVarTok}[1]{\textcolor[rgb]{0.56,0.35,0.01}{\textbf{\textit{#1}}}}
\newcommand{\ConstantTok}[1]{\textcolor[rgb]{0.00,0.00,0.00}{#1}}
\newcommand{\ControlFlowTok}[1]{\textcolor[rgb]{0.13,0.29,0.53}{\textbf{#1}}}
\newcommand{\DataTypeTok}[1]{\textcolor[rgb]{0.13,0.29,0.53}{#1}}
\newcommand{\DecValTok}[1]{\textcolor[rgb]{0.00,0.00,0.81}{#1}}
\newcommand{\DocumentationTok}[1]{\textcolor[rgb]{0.56,0.35,0.01}{\textbf{\textit{#1}}}}
\newcommand{\ErrorTok}[1]{\textcolor[rgb]{0.64,0.00,0.00}{\textbf{#1}}}
\newcommand{\ExtensionTok}[1]{#1}
\newcommand{\FloatTok}[1]{\textcolor[rgb]{0.00,0.00,0.81}{#1}}
\newcommand{\FunctionTok}[1]{\textcolor[rgb]{0.00,0.00,0.00}{#1}}
\newcommand{\ImportTok}[1]{#1}
\newcommand{\InformationTok}[1]{\textcolor[rgb]{0.56,0.35,0.01}{\textbf{\textit{#1}}}}
\newcommand{\KeywordTok}[1]{\textcolor[rgb]{0.13,0.29,0.53}{\textbf{#1}}}
\newcommand{\NormalTok}[1]{#1}
\newcommand{\OperatorTok}[1]{\textcolor[rgb]{0.81,0.36,0.00}{\textbf{#1}}}
\newcommand{\OtherTok}[1]{\textcolor[rgb]{0.56,0.35,0.01}{#1}}
\newcommand{\PreprocessorTok}[1]{\textcolor[rgb]{0.56,0.35,0.01}{\textit{#1}}}
\newcommand{\RegionMarkerTok}[1]{#1}
\newcommand{\SpecialCharTok}[1]{\textcolor[rgb]{0.00,0.00,0.00}{#1}}
\newcommand{\SpecialStringTok}[1]{\textcolor[rgb]{0.31,0.60,0.02}{#1}}
\newcommand{\StringTok}[1]{\textcolor[rgb]{0.31,0.60,0.02}{#1}}
\newcommand{\VariableTok}[1]{\textcolor[rgb]{0.00,0.00,0.00}{#1}}
\newcommand{\VerbatimStringTok}[1]{\textcolor[rgb]{0.31,0.60,0.02}{#1}}
\newcommand{\WarningTok}[1]{\textcolor[rgb]{0.56,0.35,0.01}{\textbf{\textit{#1}}}}
\usepackage{longtable,booktabs}
% Fix footnotes in tables (requires footnote package)
\IfFileExists{footnote.sty}{\usepackage{footnote}\makesavenoteenv{longtable}}{}
\usepackage{graphicx,grffile}
\makeatletter
\def\maxwidth{\ifdim\Gin@nat@width>\linewidth\linewidth\else\Gin@nat@width\fi}
\def\maxheight{\ifdim\Gin@nat@height>\textheight\textheight\else\Gin@nat@height\fi}
\makeatother
% Scale images if necessary, so that they will not overflow the page
% margins by default, and it is still possible to overwrite the defaults
% using explicit options in \includegraphics[width, height, ...]{}
\setkeys{Gin}{width=\maxwidth,height=\maxheight,keepaspectratio}
\setlength{\emergencystretch}{3em}  % prevent overfull lines
\providecommand{\tightlist}{%
  \setlength{\itemsep}{0pt}\setlength{\parskip}{0pt}}
\setcounter{secnumdepth}{0}
% Redefines (sub)paragraphs to behave more like sections
\ifx\paragraph\undefined\else
\let\oldparagraph\paragraph
\renewcommand{\paragraph}[1]{\oldparagraph{#1}\mbox{}}
\fi
\ifx\subparagraph\undefined\else
\let\oldsubparagraph\subparagraph
\renewcommand{\subparagraph}[1]{\oldsubparagraph{#1}\mbox{}}
\fi

% set default figure placement to htbp
\makeatletter
\def\fps@figure{htbp}
\makeatother


\title{Repro. Research Project 1}
\author{slauzon}
\date{10/4/2020}

\begin{document}
\maketitle

\#\#Set up

\begin{Shaded}
\begin{Highlighting}[]
\NormalTok{knitr}\OperatorTok{::}\NormalTok{opts_chunk}\OperatorTok{$}\KeywordTok{set}\NormalTok{(}\DataTypeTok{echo =} \OtherTok{TRUE}\NormalTok{, }\DataTypeTok{warning =} \OtherTok{FALSE}\NormalTok{, }\DataTypeTok{fig.width =} \DecValTok{10}\NormalTok{, }\DataTypeTok{fig.height =} \DecValTok{5}\NormalTok{, }\DataTypeTok{fig.keep =} \StringTok{'all'}\NormalTok{, }\DataTypeTok{fig.path =} \StringTok{'figures\textbackslash{} '}\NormalTok{, }\DataTypeTok{dev =} \StringTok{'png'}\NormalTok{)}
\end{Highlighting}
\end{Shaded}

\hypertarget{loading-and-preprocessing-the-data}{%
\subsection{Loading and preprocessing the
data}\label{loading-and-preprocessing-the-data}}

\hypertarget{unzip-data-to-obtain-a-csv-file.}{%
\subsection{Unzip data to obtain a csv
file.}\label{unzip-data-to-obtain-a-csv-file.}}

\begin{Shaded}
\begin{Highlighting}[]
\KeywordTok{library}\NormalTok{(}\StringTok{"data.table"}\NormalTok{)}
\KeywordTok{library}\NormalTok{(ggplot2)}
\NormalTok{fileUrl <-}\StringTok{ "https://d396qusza40orc.cloudfront.net/repdata%2Fdata%2Factivity.zip"}
\KeywordTok{download.file}\NormalTok{(fileUrl, }\DataTypeTok{destfile =} \KeywordTok{paste0}\NormalTok{(}\KeywordTok{getwd}\NormalTok{(), }\StringTok{'/repdata%2Fdata%2Factivity.zip'}\NormalTok{), }\DataTypeTok{method =} \StringTok{"curl"}\NormalTok{)}
\KeywordTok{unzip}\NormalTok{(}\StringTok{"repdata%2Fdata%2Factivity.zip"}\NormalTok{,}\DataTypeTok{exdir =} \StringTok{"PA1_template_files"}\NormalTok{)}
\end{Highlighting}
\end{Shaded}

\hypertarget{reading-csv-data-into-data.table.}{%
\subsection{Reading csv Data into
Data.Table.}\label{reading-csv-data-into-data.table.}}

\begin{Shaded}
\begin{Highlighting}[]
\NormalTok{activityDT <-}\StringTok{ }\NormalTok{data.table}\OperatorTok{::}\KeywordTok{fread}\NormalTok{(}\DataTypeTok{input =} \StringTok{"PA1_template_files/activity.csv"}\NormalTok{)}
\end{Highlighting}
\end{Shaded}

\#\#1. Calculate the total number of steps taken per day

\begin{Shaded}
\begin{Highlighting}[]
\NormalTok{Total_Steps <-}\StringTok{ }\NormalTok{activityDT[, }\KeywordTok{c}\NormalTok{(}\KeywordTok{lapply}\NormalTok{(.SD, sum, }\DataTypeTok{na.rm =} \OtherTok{FALSE}\NormalTok{)), .SDcols =}\StringTok{ }\KeywordTok{c}\NormalTok{(}\StringTok{"steps"}\NormalTok{), by =}\StringTok{ }\NormalTok{.(date)] }
\KeywordTok{head}\NormalTok{(Total_Steps, }\DecValTok{10}\NormalTok{)}
\end{Highlighting}
\end{Shaded}

\begin{verbatim}
##           date steps
##  1: 2012-10-01    NA
##  2: 2012-10-02   126
##  3: 2012-10-03 11352
##  4: 2012-10-04 12116
##  5: 2012-10-05 13294
##  6: 2012-10-06 15420
##  7: 2012-10-07 11015
##  8: 2012-10-08    NA
##  9: 2012-10-09 12811
## 10: 2012-10-10  9900
\end{verbatim}

\includegraphics{figures unnamed-chunk-4-1.png}

\hypertarget{calculate-and-report-the-mean-and-median-of-the-total-number-of-steps-taken-per-day}{%
\subsection{Calculate and report the mean and median of the total number
of steps taken per
day}\label{calculate-and-report-the-mean-and-median-of-the-total-number-of-steps-taken-per-day}}

\begin{Shaded}
\begin{Highlighting}[]
\NormalTok{Total_Steps[, .(}\DataTypeTok{Mean_Steps =} \KeywordTok{mean}\NormalTok{(steps, }\DataTypeTok{na.rm =} \OtherTok{TRUE}\NormalTok{), }\DataTypeTok{Median_Steps =} \KeywordTok{median}\NormalTok{(steps, }\DataTypeTok{na.rm =} \OtherTok{TRUE}\NormalTok{))]}
\end{Highlighting}
\end{Shaded}

\begin{verbatim}
##    Mean_Steps Median_Steps
## 1:   10766.19        10765
\end{verbatim}

\#\#1. Make a time series plot (i.e.~𝚝𝚢𝚙𝚎 = ``𝚕'') of the 5-minute
interval (x-axis) and the average number of steps taken, averaged across
all days (y-axis)

\begin{Shaded}
\begin{Highlighting}[]
\NormalTok{IntervalDT <-}\StringTok{ }\NormalTok{activityDT[, }\KeywordTok{c}\NormalTok{(}\KeywordTok{lapply}\NormalTok{(.SD, mean, }\DataTypeTok{na.rm =} \OtherTok{TRUE}\NormalTok{)), .SDcols =}\StringTok{ }\KeywordTok{c}\NormalTok{(}\StringTok{"steps"}\NormalTok{), by =}\StringTok{ }\NormalTok{.(interval)] }
\KeywordTok{ggplot}\NormalTok{(IntervalDT, }\KeywordTok{aes}\NormalTok{(}\DataTypeTok{x =}\NormalTok{ interval , }\DataTypeTok{y =}\NormalTok{ steps)) }\OperatorTok{+}\StringTok{ }\KeywordTok{geom_line}\NormalTok{(}\DataTypeTok{color=}\StringTok{"blue"}\NormalTok{, }\DataTypeTok{size=}\DecValTok{1}\NormalTok{) }\OperatorTok{+}\StringTok{ }\KeywordTok{labs}\NormalTok{(}\DataTypeTok{title =} \StringTok{"Avg. Daily Steps"}\NormalTok{, }\DataTypeTok{x =} \StringTok{"Interval"}\NormalTok{, }\DataTypeTok{y =} \StringTok{"Avg. Steps per day"}\NormalTok{)}
\end{Highlighting}
\end{Shaded}

\includegraphics{figures unnamed-chunk-6-1.png}

\#\#2. Which 5-minute interval, on average across all the days in the
dataset, contains the maximum number of steps?

\begin{Shaded}
\begin{Highlighting}[]
\NormalTok{IntervalDT[steps }\OperatorTok{==}\StringTok{ }\KeywordTok{max}\NormalTok{(steps), .(}\DataTypeTok{max_interval =}\NormalTok{ interval)]}
\end{Highlighting}
\end{Shaded}

\begin{verbatim}
##    max_interval
## 1:          835
\end{verbatim}

\#\#1. Calculate and report the total number of missing values in the
dataset (i.e.~the total number of rows with 𝙽𝙰s)

\begin{Shaded}
\begin{Highlighting}[]
\NormalTok{activityDT[}\KeywordTok{is.na}\NormalTok{(steps), .N ]}
\end{Highlighting}
\end{Shaded}

\begin{verbatim}
## [1] 2304
\end{verbatim}

\#\#2. Devise a strategy for filling in all of the missing values in the
dataset. The strategy does not need to be sophisticated. For example,
you could use the mean/median for that day, or the mean for that
5-minute interval, etc.

\begin{Shaded}
\begin{Highlighting}[]
\CommentTok{# Filling in missing values with median of dataset. }
\NormalTok{activityDT[}\KeywordTok{is.na}\NormalTok{(steps), }\StringTok{"steps"}\NormalTok{] <-}\StringTok{ }\NormalTok{activityDT[, }\KeywordTok{c}\NormalTok{(}\KeywordTok{lapply}\NormalTok{(.SD, median, }\DataTypeTok{na.rm =} \OtherTok{TRUE}\NormalTok{)), .SDcols =}\StringTok{ }\KeywordTok{c}\NormalTok{(}\StringTok{"steps"}\NormalTok{)]}
\end{Highlighting}
\end{Shaded}

\#\#3. Create a new dataset that is equal to the original dataset but
with the missing data filled in.

\begin{Shaded}
\begin{Highlighting}[]
\NormalTok{data.table}\OperatorTok{::}\KeywordTok{fwrite}\NormalTok{(}\DataTypeTok{x =}\NormalTok{ activityDT, }\DataTypeTok{file =} \StringTok{"PA1_template_files/Data.csv"}\NormalTok{, }\DataTypeTok{quote =} \OtherTok{FALSE}\NormalTok{)}
\end{Highlighting}
\end{Shaded}

\#\#4. Make a histogram of the total number of steps taken each day and
calculate and report the mean and median total number of steps taken per
day. Do these values differ from the estimates from the first part of
the assignment? What is the impact of imputing missing data on the
estimates of the total daily number of steps?

\begin{Shaded}
\begin{Highlighting}[]
\CommentTok{# total number of steps taken per day}
\NormalTok{Total_Steps <-}\StringTok{ }\NormalTok{activityDT[, }\KeywordTok{c}\NormalTok{(}\KeywordTok{lapply}\NormalTok{(.SD, sum)), .SDcols =}\StringTok{ }\KeywordTok{c}\NormalTok{(}\StringTok{"steps"}\NormalTok{), by =}\StringTok{ }\NormalTok{.(date)] }
\CommentTok{# mean and median total number of steps taken per day}
\NormalTok{Total_Steps[, .(}\DataTypeTok{Mean_Steps =} \KeywordTok{mean}\NormalTok{(steps), }\DataTypeTok{Median_Steps =} \KeywordTok{median}\NormalTok{(steps))]}
\end{Highlighting}
\end{Shaded}

\begin{verbatim}
##    Mean_Steps Median_Steps
## 1:    9354.23        10395
\end{verbatim}

\begin{Shaded}
\begin{Highlighting}[]
\KeywordTok{ggplot}\NormalTok{(Total_Steps, }\KeywordTok{aes}\NormalTok{(}\DataTypeTok{x =}\NormalTok{ steps)) }\OperatorTok{+}\StringTok{ }\KeywordTok{geom_histogram}\NormalTok{(}\DataTypeTok{fill =} \StringTok{"blue"}\NormalTok{, }\DataTypeTok{binwidth =} \DecValTok{1000}\NormalTok{) }\OperatorTok{+}\StringTok{ }\KeywordTok{labs}\NormalTok{(}\DataTypeTok{title =} \StringTok{"Daily Steps"}\NormalTok{, }\DataTypeTok{x =} \StringTok{"Steps"}\NormalTok{, }\DataTypeTok{y =} \StringTok{"Frequency"}\NormalTok{)}
\end{Highlighting}
\end{Shaded}

\includegraphics{figures unnamed-chunk-11-1.png}

\begin{longtable}[]{@{}lll@{}}
\toprule
Type of Estimate & Mean\_Steps & Median\_Steps\tabularnewline
\midrule
\endhead
First Part (with na) & 10765 & 10765\tabularnewline
Second Part (fillin in na with median) & 9354.23 & 10395\tabularnewline
\bottomrule
\end{longtable}

\#\#1. Create a new factor variable in the dataset with two levels --
``weekday'' and ``weekend'' indicating whether a given date is a weekday
or weekend day.

\begin{Shaded}
\begin{Highlighting}[]
\CommentTok{# Just recreating activityDT from scratch then making the new factor variable. (No need to, just want to be clear on what the entire process is.) }
\NormalTok{activityDT <-}\StringTok{ }\NormalTok{data.table}\OperatorTok{::}\KeywordTok{fread}\NormalTok{(}\DataTypeTok{input =} \StringTok{"PA1_template_files/activity.csv"}\NormalTok{)}
\NormalTok{activityDT[, date }\OperatorTok{:}\ErrorTok{=}\StringTok{ }\KeywordTok{as.POSIXct}\NormalTok{(date, }\DataTypeTok{format =} \StringTok{"%Y-%m-%d"}\NormalTok{)]}
\NormalTok{activityDT[, }\StringTok{`}\DataTypeTok{Day of Week}\StringTok{`}\OperatorTok{:}\ErrorTok{=}\StringTok{ }\KeywordTok{weekdays}\NormalTok{(}\DataTypeTok{x =}\NormalTok{ date)]}
\NormalTok{activityDT[}\KeywordTok{grepl}\NormalTok{(}\DataTypeTok{pattern =} \StringTok{"Monday|Tuesday|Wednesday|Thursday|Friday"}\NormalTok{, }\DataTypeTok{x =} \StringTok{`}\DataTypeTok{Day of Week}\StringTok{`}\NormalTok{), }\StringTok{"weekday or weekend"}\NormalTok{] <-}\StringTok{ "weekday"}
\NormalTok{activityDT[}\KeywordTok{grepl}\NormalTok{(}\DataTypeTok{pattern =} \StringTok{"Saturday|Sunday"}\NormalTok{, }\DataTypeTok{x =} \StringTok{`}\DataTypeTok{Day of Week}\StringTok{`}\NormalTok{), }\StringTok{"weekday or weekend"}\NormalTok{] <-}\StringTok{ "weekend"}
\NormalTok{activityDT[, }\StringTok{`}\DataTypeTok{weekday or weekend}\StringTok{`} \OperatorTok{:}\ErrorTok{=}\StringTok{ }\KeywordTok{as.factor}\NormalTok{(}\StringTok{`}\DataTypeTok{weekday or weekend}\StringTok{`}\NormalTok{)]}
\KeywordTok{head}\NormalTok{(activityDT, }\DecValTok{10}\NormalTok{)}
\end{Highlighting}
\end{Shaded}

\begin{verbatim}
##     steps       date interval Day of Week weekday or weekend
##  1:    NA 2012-10-01        0      Monday            weekday
##  2:    NA 2012-10-01        5      Monday            weekday
##  3:    NA 2012-10-01       10      Monday            weekday
##  4:    NA 2012-10-01       15      Monday            weekday
##  5:    NA 2012-10-01       20      Monday            weekday
##  6:    NA 2012-10-01       25      Monday            weekday
##  7:    NA 2012-10-01       30      Monday            weekday
##  8:    NA 2012-10-01       35      Monday            weekday
##  9:    NA 2012-10-01       40      Monday            weekday
## 10:    NA 2012-10-01       45      Monday            weekday
\end{verbatim}

\#\#2. Make a panel plot containing a time series plot (i.e.~𝚝𝚢𝚙𝚎 =
``𝚕'') of the 5-minute interval (x-axis) and the average number of steps
taken, averaged across all weekday days or weekend days (y-axis). See
the README file in the GitHub repository to see an example of what this
plot should look like using simulated data.

\begin{Shaded}
\begin{Highlighting}[]
\NormalTok{activityDT[}\KeywordTok{is.na}\NormalTok{(steps), }\StringTok{"steps"}\NormalTok{] <-}\StringTok{ }\NormalTok{activityDT[, }\KeywordTok{c}\NormalTok{(}\KeywordTok{lapply}\NormalTok{(.SD, median, }\DataTypeTok{na.rm =} \OtherTok{TRUE}\NormalTok{)), .SDcols =}\StringTok{ }\KeywordTok{c}\NormalTok{(}\StringTok{"steps"}\NormalTok{)]}
\NormalTok{IntervalDT <-}\StringTok{ }\NormalTok{activityDT[, }\KeywordTok{c}\NormalTok{(}\KeywordTok{lapply}\NormalTok{(.SD, mean, }\DataTypeTok{na.rm =} \OtherTok{TRUE}\NormalTok{)), .SDcols =}\StringTok{ }\KeywordTok{c}\NormalTok{(}\StringTok{"steps"}\NormalTok{), by =}\StringTok{ }\NormalTok{.(interval, }\StringTok{`}\DataTypeTok{weekday or weekend}\StringTok{`}\NormalTok{)] }
\KeywordTok{ggplot}\NormalTok{(IntervalDT , }\KeywordTok{aes}\NormalTok{(}\DataTypeTok{x =}\NormalTok{ interval , }\DataTypeTok{y =}\NormalTok{ steps, }\DataTypeTok{color=}\StringTok{`}\DataTypeTok{weekday or weekend}\StringTok{`}\NormalTok{)) }\OperatorTok{+}\StringTok{ }\KeywordTok{geom_line}\NormalTok{() }\OperatorTok{+}\StringTok{ }\KeywordTok{labs}\NormalTok{(}\DataTypeTok{title =} \StringTok{"Avg. Daily Steps by Weektype"}\NormalTok{, }\DataTypeTok{x =} \StringTok{"Interval"}\NormalTok{, }\DataTypeTok{y =} \StringTok{"No. of Steps"}\NormalTok{) }\OperatorTok{+}\StringTok{ }\KeywordTok{facet_wrap}\NormalTok{(}\OperatorTok{~}\StringTok{`}\DataTypeTok{weekday or weekend}\StringTok{`}\NormalTok{ , }\DataTypeTok{ncol =} \DecValTok{1}\NormalTok{, }\DataTypeTok{nrow=}\DecValTok{2}\NormalTok{)}
\end{Highlighting}
\end{Shaded}

\includegraphics{figures unnamed-chunk-13-1.png}

\end{document}
